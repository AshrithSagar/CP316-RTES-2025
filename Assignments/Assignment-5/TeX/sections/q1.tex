\section*{Part 1: Context switch}

\subsection*{Question 1}

The code mentiones `hardware context' – what is it?

\subsubsection*{Solution}

The hardware context refers to the CPU registers that must be saved and restored when switching between tasks.
This includes registers that the CPU automatically saves on exceptions, as well as registers that the OS must manually save and restore.
When an exception occurs, the CPU automatically pushes the following onto the stack: \texttt{xPSR}, \texttt{PC}, \texttt{LR}, \texttt{R12}, \texttt{R3}, \texttt{R2}, \texttt{R1}, \texttt{R0}.
In addition, the OS must explicitly save registers \texttt{R4} to \texttt{R11}.
