\section*{Part 1}

Timing measurements

Program SysTick with a maximum reload value of \hex{0x00FFFFFF} and use CVR for timing measurements.

Measure the following:
\begin{enumerate}[label= (\alph*), noitemsep, topsep=0pt]
    \item Function-call and loop overheads, and clock cycles per iteration in \texttt{delay\@()} function.
    \item Time spent to transmit one character.
    \item Average human reaction-time to click (press and release) a button.
\end{enumerate}

You may experiment with different loop sizes, multiple characters and mutliple switch presses to arrive at a ``reasonably'' accurate result.

\subsection*{Solution}

\subsubsection*{Part 1a}

We make use of two timers here, one to measure the function overhead, which overflows every \texttt{1\,ms}, and one to show the averages over UART, which overflows every \texttt{1000\,ms}.
At the core of the \texttt{measure\@()} function, we use a \texttt{delay\_ms\@(1)} with argument \texttt{1} to measure the time taken for a single loop iteration.
The timestamps are handled by reading the \texttt{CVR} register of the SysTick timer, accordingly.

A sample from the log displayed:
\begin{verbatim}
Function call overhead: Average: 6 microseconds, across 1000 iterations
Loop overhead: Average: 95.98 microseconds, across 1000 iterations
Clock cycles per iteration: 6147 clock cycles
\end{verbatim}

\subsubsection*{Part 1b}

Similarly, to find the time spent to transmit one character, we configure the timers same as above, but the core is changed to use \texttt{uart\_putc(' ');} transmitting a single space character.

Sample log:
\begin{verbatim}
Average: 83.44 microseconds, across 1000 iterations
\end{verbatim}

\subsubsection*{Part 1c}

To measure the average human reaction time, here we now use three timers, one to turn the LED on every \texttt{1000\,ms}, and run a reaction listener every \texttt{1\,ms}, and one to show the averages over UART, which overflows every \texttt{10000\,ms}.

A sample from the log displayed:
\begin{verbatim}
Average: 0.10 microseconds, across 10 iterations
\end{verbatim}
which doesn't seem right, the fix of which I wasn't able to find.

\section*{Part 2}

Implementation deviates a bit from the instructions.
Instead of having separate button presses for the start and stop of the stopwatch, holding the button down starts the stopwatch, and releasing it stops it.
