\section*{Question}

Read Section 9 GPIOTE in \texttt{nRF52833\_v1.5} product specifications.
Also refer to the switch interfacing circuit in the schematics.

Explain the sequence from the button-press till interrupt handler returns and main code resumes.

\vspace*{-2em}
\subsection*{Solution}

GPIOTE stands for General Purpose Input/Output Tasks and Events, which is a module for providing functionality for accessing the tasks and events of the GPIO pins.
It can be used to generate events when the state of a GPIO pin changes as well as to control the state of GPIO pins.

Here, we use it to detect the button press and accordingly toggle an LED, by enabling interrupts for the falling edge event for \texttt{BUTTON\_0}.
GPIOTE has 8 registers, each which can be configured for some event (number) \( i \) [\( i \to 0 \text{ to } 7 \)] along with the pin number to track and event type [rising edge / falling edge / any change] associated with the event.
The \texttt{GPIOTE\_INTENSET} register holds the interrupt enable bits for each of these events, which we have to set the corresponding bit to 1 to enable the interrupt for that event.
Corresponding to each event is a \texttt{GPIOTE\_CONFIG} register, which holds the configuration for that event, which we configure as follows (Figure~\ref{fig:gpiote-config}):
\begin{itemize}[noitemsep, topsep=0pt]
    \item The mode is set to \texttt{Event} mode, by writing \texttt{01} in the \texttt{MODE} field, i.e., in the \texttt{AA} bits.
          This allows the pin specified in \texttt{PSEL} to be configured as an input and an input event is generated whenever the edge type operation occurs in the pin.

    \item The pin number is set in the \texttt{PSEL} field, i.e., in the \texttt{BBBBB} bits.
          \texttt{BUTTON\_0} is connected to pin 14, so we set \texttt{01110} here.

    \item The edge type is set in the \texttt{POLARITY} field, i.e., in the \texttt{C} bit.
          For falling edge detection, we set it to 2, i.e., as \texttt{10}.
\end{itemize}

\begin{figure}[h]
    \centering
    \includegraphics[width=\textwidth]{figures/q0/_}
    \vspace*{-3.5em}
    \caption{
        \texttt{GPIOTE\_CONFIG} register for a GPIOTE event.
    }\label{fig:gpiote-config}
\end{figure}

Additionally, we have to enable GPIOTE interrupts in the interrupt controller by setting the \texttt{GPIOTE} bit (\texttt{bit-6}) in the \texttt{NVIC\_ISER} register.

Once the button is pressed, the falling edge event is detected for it, and an interrupt is generated, changing the execution flow to the interrupt handler.
The execution flow now calls the corresponding callback function, which in this case is the \texttt{button0\_press} function, which toggles the LED state accordingly.
Once this is done, the interrupt handler returns, and the main code resumes execution from where it was interrupted.
