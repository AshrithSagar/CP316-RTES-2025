\section*{Question 2}

How is a frequency sweep generated?

\subsubsection*{Solution}

A frequency sweep is produced using the function \texttt{audio\_sweep\@(start\_freq, end\_freq, duration\_ms)}, which works by generating many linearly interpolated beeps from the start frequency to the end frequency over the specified duration in milliseconds.
Each of it's constituent beep is generated by the \texttt{audio\_beep\@(freq, duration\_ms)} which plays a uniform beep at the specified frequency for the specified duration.
By linearly interpolating the frequency from \texttt{start\_freq} to \texttt{end\_freq} over the duration, we're able to create a smooth sweep effect.
Internally, this uses the \texttt{PWM} registers that \texttt{nRF52} provides to do this.
