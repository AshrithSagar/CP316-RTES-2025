\section*{Question 1}

Explain the objective and functionality of \texttt{led\_row\_refresh} function.

\begin{enumerate}[label= (\alph*), noitemsep, topsep=0pt]
    \item How frequently it is called? Why?
    \item Who calls it?
    \item What happens if the processor is stuck into a delay loop when the display need to be refreshed?
\end{enumerate}

\subsubsection*{Solution}

The \texttt{led\_row\_refresh\@()} routine is responsible for refreshing the LED display.
Each time it is called, it clears the pixels of the previous row and sets the pixels of the current row.
It is called every \texttt{5\,\texttt{ms}} by the \texttt{SysTick\_Handler} interrupt service routine (ISR).
With 5 rows in the display, once cycle of iteration through the display takes \( 5 \times 5 = \) \texttt{25\,\texttt{ms}} \( \implies \frac{1000}{25} = 40\,\texttt{Hz} \) refresh rate, which is sufficient for the human eye to perceive the display as continuous without flickering.

When the display needs to be refreshed, i.e., as and when the timer interrupt occurs, the control shifts to the \texttt{SysTick\_Handler} ISR due to this interrupt, which happens at a hardware level.
The handler runs the \texttt{led\_row\_refresh} function, which is designed to have a minimal overhead, after which the control returns to the main program, and we continue executing what we were doing before the interrupt.
If the processor is stuck in a delay loop when the display needs to be refreshed, the interrupt triggers the display refresh, and upon returning to the main flow, the processor will continue to wait from where it was stuck.
