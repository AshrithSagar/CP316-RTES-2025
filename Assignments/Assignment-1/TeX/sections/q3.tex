\section*{Question 3}

Understanding header files:
Add the following line in `build.sh`:
\begin{verbatim}
$CC -E $CFLAGS main.c > main.i
\end{verbatim}
\begin{itemize}[noitemsep]
    \item What does `\texttt{-E}' do?
    \item How many lines are there in \texttt{main.i}? Where did these lines come from?
    \item Functions in \texttt{myled.h} are defined as \texttt{static inline}.
          What does \texttt{static inline} mean?
\end{itemize}

\subsection*{Solution}

The `\texttt{-E}` option indicates to just return the preprocessed source before the actual compilation.
This preprocessed source is redirected to be stored in the file \texttt{main.i}.
The contents of this \texttt{main.i} file are very similar to the file \texttt{main.c} but with all the preprocessor directives expanded, the comments removed, and the macros replaced with their definitions.
The number of lines in the \texttt{main.i} file is 164, which is much greater than the same in the \texttt{main.c} file, which has 33 lines.

The keyword \texttt{static} in C is used to declare a function or a variable that is only visible to the file in which it is declared.
The keyword \texttt{inline} is used to suggest the compiler to insert the code of the function in place of the function call, i.e., to perform an inline substitution of the function call.
This is optional for the compiler to follow, and is just a suggestion and not a hard rule.
Hence, the \texttt{static inline} keyword combination is used to define a function that is only visible to the file in which it is declared, in this case, \texttt{myled.h}, and to suggest the compiler to perform an inline substitution of the function call, which is a good optimization technique for small functions.
