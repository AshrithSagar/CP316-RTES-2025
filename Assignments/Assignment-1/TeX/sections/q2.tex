\section*{Question 2}

What is an executable program?
How is it generated?

\subsection*{Solution}

An executable program is a file that contains the machine code of a program that can be executed by the operating system.
It is generated by linking one or more object files together by the linker.
The executable file has an extension that is specific to the operating system, such as \texttt{.out} for Unix-like systems and \texttt{.exe} for Windows.
In the code given here, the executable file \texttt{tiny.out} is generated by linking the object files \texttt{startup.o}, \texttt{system.o}, \texttt{main.o}, and \texttt{uart.o} together using the following command:
\begin{verbatim}
arm-none-eabi-ld -T bare.ld -Map tiny.map \
    startup.o system.o main.o uart.o -o tiny.out
\end{verbatim}
