\section*{Part C}

Define the memory management scheme for the audio buffers, using statically or dynamically allocated buffers.

\subsection*{Solution}

We can choose a dual-buffer (ping-pong) scheme with statically allocated memory to avoid fragmentation and ensure real-time determinism.

\begin{lstlisting}[language=C]
#define AUDIO_WINDOW_SAMPLES 256
// Two contiguous buffers in SRAM to avoid dynamic allocation overhead
static int16_t audioBuf[2][AUDIO_WINDOW_SAMPLES];
static volatile uint8_t prodIdx = 0;
static volatile uint8_t consIdx = 1;
// Semaphore signaled by ISR when prod buffer full
SemaphoreHandle_t audioSem;

void swapBuffers(void) {
    uint8_t temp = prodIdx;
    prodIdx = consIdx;
    consIdx = temp;
}
\end{lstlisting}

\noindent
Buffer swap occurs entirely in the ISR context within a critical section (\( < 5 \) \,µs), guaranteeing that the AudioProc Task always sees a complete window.
