\section*{Part C}

Define the memory management scheme for the audio buffers, using statically or dynamically allocated buffers.

\subsection*{Solution}

We can use a statically allocated buffer since it's simpler to implement here.
In addition, we go with a double buffering scheme.
The \texttt{ADC ISR} writes the incoming samples to one buffer while the voice command detection algorithm processes the other buffer.
Once a full window of samples are available, the \texttt{ISR} can signal a semaphore to notify the processing task, after which the processing task can swap the buffers and start processing the new buffer while the \texttt{ISR} continues to fill the other buffer.
We just need a double buffer here, since the processing time of the detection algorithm, given that it takes about \texttt{8\,ms} to \texttt{12\,ms} to execute, is less than the time taken to fill a buffer of \texttt{32\,ms} duration.

The sizes required for the buffers can be calculated as follows:
The \texttt{ADC} samples at \( 8\,\texttt{kHz} \) for \( 32\,\texttt{ms} \), so the number of samples per window is \( 8\,\texttt{kHz} \times 32\,\texttt{ms} \), i.e., 256 samples per window.
With each sample being \( 16\,\texttt{bits} \implies 2\,\texttt{bytes} \), the size needed for each buffer is then \( 256 \times 2 = 512\,\texttt{bytes} \), thereby the total size of the two buffers would be \( 512 \times 2 = 1024\,\texttt{bytes} \).
