\section*{Instructions}

This is a take-home exam.
It should take about 3 hours.
Spending about an hour in part (A), understanding the functionality, dividing it into various tasks and ISRs and thinking through the key issues will help.

Subsequently, spend 20--30 minutes on each of B, C and D.
Finally, complete your answers with E.
Don’t worry about getting everything perfect.
Ensure that the key aspects of the design are covered and presented properly.

You may make reasonable assumptions and state them clearly.
Feel free to reach out for clarifications or discussions.

\vspace*{-1.5em}
\subsection*{System Scenario}
\vspace*{-0.5em}

You are designing embedded software for a small autonomous robot with line-following, obstacle avoidance, audio interaction and debugging capabilities.
The system runs on a microcontroller with an RTOS.\@

\vspace*{-1.5em}
\subsection*{Hardware and Interfaces}
\vspace*{-0.5em}

\begin{itemize}[topsep=0pt]
      \item Eight binary light sensors for line following (high for light, low for dark)

      \item Ultrasonic distance sensor (front-facing)

      \item Two motors (left and right)

      \item Microphone connected via ADC

      \item Radio link (for debug messages from tasks)
\end{itemize}

\vspace*{-1.5em}
\subsection*{System Behavior}
\vspace*{-0.5em}

\begin{enumerate}[topsep=0pt]
      \item On power-up, the system waits in idle mode.

      \item On a valid voice command, the robot starts moving and follows the line using the light sensors.
            You don't need to implement the control algorithm for line-following; assume a black-box function is available that takes sensor input and returns motor duty cycles.

      \item If an obstacle is detected within \texttt{30\,cm}, the robot stops and waits until the obstacle is cleared before resuming line-following.

      \item The robot listens for voice commands through the microphone.
            A start word (such as ``go'') starts the robot, and a stop word (such as ``stop'') halts its operation and returns it to idle mode.

      \item The microphone is connected to a \texttt{16}-bit ADC that triggers an interrupt at \texttt{8\,kHz} sampling rate.

      \item All tasks use the radio link for debugging.
            Debug messages can be printed into a buffer (\texttt{sprintf}) and then the buffer can be transmitted over the radio interface.
\end{enumerate}

\vspace*{-1.5em}
\subsection*{Voice Command Detection Details}
\vspace*{-0.5em}

The algorithm checks for known words using a lightweight pattern-matching technique.

\begin{itemize}[topsep=0pt]
      \item The detection algorithm is implemented as a black-box function that takes in a window (buffer) of audio samples, updates its internal state, and returns a decision: index of the word detected, or \texttt{-1} for detection in progress.

      \item The algorithm runs window by window, where each window is of \texttt{32\,ms} duration.
            Once a window is passed to the algorithm, its samples are no longer required and can be discarded or reused.

      \item The algorithm takes about \texttt{8\,ms} to \texttt{12\,ms} to execute.

      \item The algorithm requires a minimum of 4 consecutive windows to make a decision.
\end{itemize}

\vspace*{-1.5em}
\subsection*{Answer the following}
\vspace*{-0.5em}

Design the system as a collection of ISRs, timer callbacks and tasks.
You also need to include mutual exclusion, synchronization and memory management objects as required.

You may assume availability of the following RTOS objects: tasks (with priorities), semaphores (binary and counting), mutexes, message queues, memory pools and timers.

\texttt{C}-like pseudocode is acceptable; the exact syntax is not important.
