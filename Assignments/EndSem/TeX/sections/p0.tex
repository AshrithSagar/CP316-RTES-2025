\section*{Instructions}

\vspace*{-1.5em}
\subsection*{System Scenario}
\vspace*{-0.5em}

You are designing embedded software for a small autonomous robot with line-following, obstacle avoidance, audio interaction and debugging capabilities.
The system runs on a microcontroller with an RTOS.\@

\vspace*{-1.5em}
\subsection*{Hardware and Interfaces}
\vspace*{-0.5em}

\begin{itemize}[noitemsep, topsep=0pt]
    \item Eight binary light sensors for line following (high for light, low for dark)
    \item Ultrasonic distance sensor (front-facing)
    \item Two motors (left and right)
    \item Microphone connected via ADC
    \item Radio link (for debug messages from tasks)
\end{itemize}

\vspace*{-1.5em}
\subsection*{System Behavior}
\vspace*{-0.5em}

\begin{enumerate}[noitemsep, topsep=0pt]
    \item On power-up, the system waits in idle mode.

    \item On a valid voice command, the robot starts moving and follows the line using the light sensors.
          You don't need to implement the control algorithm for line-following; assume a black-box function is available that takes sensor input and returns motor duty cycles.

    \item If an obstacle is detected within 30 cm, the robot stops and waits until the obstacle is cleared before resuming line-following.

    \item The robot listens for voice commands through the microphone.
          A start word (such as ``go'') starts the robot, and a stop word (such as ``stop'') halts its operation and returns it to idle mode.

    \item The microphone is connected to a 16-bit ADC that triggers an interrupt at 8 kHz sampling rate.

    \item All tasks use the radio link for debugging.
          Debug messages can be printed into a buffer (\texttt{sprintf}) and then the buffer can be transmitted over the radio interface.
\end{enumerate}

\vspace*{-1.5em}
\subsection*{Voice Command Detection Details}
\vspace*{-0.5em}

The algorithm checks for known words using a lightweight pattern-matching technique.

\begin{itemize}[noitemsep, topsep=0pt]
    \item The detection algorithm is implemented as a black-box function that takes in a window (buffer) of audio samples, updates its internal state, and returns a decision: index of the word detected, or -1 for detection in progress.

    \item The algorithm runs window by window, where each window is of 32 ms duration.
          Once a window is passed to the algorithm, its samples are no longer required and can be discarded or reused.

    \item The algorithm takes about 8ms to 12ms to execute.

    \item The algorithm requires a minimum of 4 consecutive windows to make a decision.
\end{itemize}
