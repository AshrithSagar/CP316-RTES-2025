\section*{Part B}

Describe synchronization mechanisms for the following:
\begin{itemize}[noitemsep, topsep=0pt]
    \item Receiving audio samples
    \item Receiving echo pulse for the ultrasonic sensor
    \item Sending debug messages over the radio link.
\end{itemize}

\subsection*{Solution}

\subsubsection*{Acquisition of audio samples}

We can use a message queue to synchronise receiving the audio samples and processing them.
A message queue effectively comprises a buffer and a semaphore.
Initally, the buffer is empty, and the counting semaphore is set to zero.
The 16-bit ADC fires interrupts at \( 8\,\texttt{kHz} \), after which a minimally blocking \texttt{ADC ISR} writes these samples to the buffer, and plays the role of the producer here.
It signals the semaphore to indicate that a new sample has been written to the buffer.
The consumer in this case is the black box voice command detection algorithm, which waits until there are enough samples in the buffer to process, which it knows by checking the semaphore.

\subsubsection*{Ultrasonic echo measurements}

A timer callback periodically triggers the ultrasonic sensor to send out a pulse.
When the echo pin goes high and the ISR starts a timer, and the system is not free to process other tasks.
When the echo pin goes low upon receiving an incoming pulse, the interrupt handler stops the timer and calculates the time taken for the echo to return, and in turn the distance to the obstacle, if any, in a separate task.

\subsubsection*{Transmission of debugging messages}
