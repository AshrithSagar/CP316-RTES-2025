\section*{Question 2 | Ultrasonic sensor}

A system uses an ultrasonic sensor to measure the distance from a surface in the front. \\
The sensor has two pins: trigger and echo.
Both are connected to GPIO pins on a microcontroller.
\begin{itemize}[noitemsep, topsep=0pt, leftmargin=*]
    \item
          Trigger is an output pin.
          We need to set it high for \( 10 \, \texttt{ms} \) to start the measurement.
          The sensor sends out a sound wave in the form of a series of pulses of ultrasonic frequency.
          The wave echoes back after hitting a surface.
          The time of echo is used to measure the distance from the surface.
    \item
          Echo is an input pin.
          It remains high for the duration of echo.
          For example, if it takes \( 10 \, \texttt{ms} \) for the sound wave to hit a surface and return, the echo pin remains high for \( 10 \, \texttt{ms} \).
\end{itemize}
Distance can be calculated by the formula: \( d = v * T / 2 \) \\
where, \( v = \text{velocity of sound} = 343 \, \texttt{m/s} \) \\
\( T = \) time taken for wave to hit the surface and return (traveling total distance \( 2d \)). \\
The pulse width is clipped at \( 24 \, \texttt{ms} \) corresponding to approximately \( 4 \, \texttt{m} \) distance.
Distances larger than \( 4 \, \texttt{m} \) are considered out of range.
Now, consider the implementation of an ultrasonic sensor read function:
\lstinputlisting[language=C, frame=single, linewidth=\linewidth]{../Codes/q2.c}
Answer the following:
\begin{itemize}[noitemsep, topsep=0pt]
    \item What real-time issues do you see in the function above?
    \item How should the function be modified so that it can be used from multiple threads?
\end{itemize}

\clearpage
\subsection*{Solution}

The provided implementation has a couple of real-time issues, mainly due to the busy-waiting loop, and it's blocking behavior.
Firstly, the function busy-waits for the echo pin, which has several drawbacks, such as the CPU being blocked and not being able to perform other tasks during this time.
This is a drawback of busy-waiting in general in real-time systems.

Secondly, there is no timeout implemented while waiting.
The busy-waiting implemented here is unbounded.
If there is a situation where the echo pin never goes high, the function will be stuck in an infinite loop, waiting for the echo to start.
Same holds true when it's waiting for the echo to go low, before calculating the distance.
This situation may occur due to various reasons, such as the sensor being faulty, or the sound wave not hitting any surface (\( 4 \, \texttt{m} \) cutoff mentioned, but not used), or in case of a multi-threaded system, if there is a context switch happening to some other thread, and we miss out on the echo pulse.

Thirdly, the function is not thread-safe.
It is possible for multiple threads to access the registers and modify, which can lead to race conditions.
This can be addressed by using a mutex to ensure mutual exclusion while accessing the sensor.

\vspace*{1em}
To adapt this function in a multi-threaded environment, we need to make a few changes, namely, including a timeout, implementing through interrupts, and using a mutex for mutual exclusion while accessing the sensor.

Timeouts can be implemented either through interrupts or through a timer.
In the case of interrupts, we can set up an interrupt handler for the echo pin, which will be triggered when the echo pin goes high or low.
This way, we can avoid busy-waiting and can perform other tasks while waiting for the echo.
The interrupt handler can set a flag when the echo pin goes high, and another flag when it goes low.
The main function can then wait for these flags to be set, and then proceed to calculate the distance.
This way, the function will not be blocked, and can be used from multiple threads.

\vspace*{1em}
A rough sketch of the modified function is as follows (\autoref{lst:bounded-busy}), which implements a bounded busy-waiting loop, and uses a mutex for accessing the sensor, and some error codes appropriately.
This is still polling based, so has the drawbacks of CPU being blocked waiting for the echo.

A rough sketch of the modified function using interrupts is as follows (\autoref{lst:interrupts}), which uses interrupts to handle the echo pin, avoiding busy-waiting altogether.
This way, the function can be used from multiple threads, and the CPU is not blocked while waiting for the echo.

\vspace*{1em}
\textbf{Additional remarks}
Assuming \texttt{get\_time\_in\_usec()} is a function that returns the current time in microseconds.
Also, not sure of how to avoid performing an \texttt{OS} call in an interrupt handler (\autoref{lst:interrupts}), although this pattern is not recommended owing to keeping interrupt handlers as short as possible.

\clearpage
\lstinputlisting[language=C, frame=single, linewidth=\linewidth, caption={Modified function using bounded busy-wait and mutex}, label={lst:bounded-busy}]{../Codes/q2a.c}

\clearpage
\lstinputlisting[language=C, frame=single, linewidth=\linewidth, caption={Modified function using interrupts in addition}, label={lst:interrupts}]{../Codes/q2b.c}
