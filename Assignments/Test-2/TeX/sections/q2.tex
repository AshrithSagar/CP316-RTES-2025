\section*{Question 2 | Ultrasonic sensor}

A system uses an ultrasonic sensor to measure the distance from a surface in the front. \\
The sensor has two pins: trigger and echo.
Both are connected to GPIO pins on a microcontroller.
\begin{itemize}[noitemsep, topsep=0pt, leftmargin=*]
    \item
          Trigger is an output pin.
          We need to set it high for \( 10 \texttt{ ms} \) to start the measurement.
          The sensor sends out a sound wave in the form of a series of pulses of ultrasonic frequency.
          The wave echoes back after hitting a surface.
          The time of echo is used to measure the distance from the surface.
    \item
          Echo is an input pin.
          It remains high for the duration of echo.
          For example, if it takes \( 10 \texttt{ ms} \) for the sound wave to hit a surface and return, the echo pin remains high for \( 10 \texttt{ ms} \).
\end{itemize}
Distance can be calculated by the formula: \( d = v * T / 2 \) \\
where, \( v = \text{velocity of sound} = 343 m/s \) \\
\( T = \text{time taken for wave to hit the surface and return (traveling total distance 2d)} \). \\
The pulse width is clipped at \( 24 \texttt{ ms} \) corresponding to approximately \( 4 \texttt{ m} \) distance.
Distances larger than \( 4 \texttt{ m} \) are considered out of range.
Now, consider the implementation of an ultrasonic sensor read function:
\lstinputlisting[language=C, frame=single, linewidth=\linewidth]{../Codes/q2.c}
Answer the following:
\begin{itemize}[noitemsep, topsep=0pt]
    \item What real-time issues do you see in the function above?
    \item How should the function be modified so that it can be used from multiple threads?
\end{itemize}

\vspace*{-1em}
\subsection*{Solution}
